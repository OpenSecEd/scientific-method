\title{%
  A Science of Security Course
}
\author{\textbf{Daniel Bosk} \and Sonja Buchegger}
\institute{%
  KTH EECS, \email{dbosk@kth.se}
}

\mode<article>{\maketitle}
\mode<presentation>{%
  \begin{frame}
    \maketitle
  \end{frame}
}

\mode*

\begin{abstract}
  \mode*

% What's the problem?
% Why is it a problem? Research gap left by other approaches?
% Why is it important? Why care?
% What's the approach? How to solve the problem?
% What's the findings? How was it evaluated, what are the results, limitations, 
% what remains to be done?

% XXX Summary
\emph{Summary:}
In this learning session we will give an introduction to the scientific method 
and particularly how this can be applied in the area of security.

% XXX Motivation and intended learning outcomes
\emph{Intended learning outcomes:}
After this session you should be able:
\begin{itemize}
  \item to \emph{apply} the scientific method correctly to answer basic 
    questions in security.
\end{itemize}

% XXX Prerequisites
%\emph{Prerequisites:}
%\dots

% XXX Reading material
\emph{Reading:}
You should read 
\citetitle{HowToDesignSecurityExperiments}~\cite{HowToDesignSecurityExperiments}.
This paper discusses the scientific method of (parts of) the security field.
For a more in-depth reflection on the state of security as a scientific pursuit, 
we recommend
\citetitle{SecurityAsAScience}~\cite{SecurityAsAScience}.

\end{abstract}


\section{Introduction}

\subsection{The origin of the problem}

\begin{frame}
  \blockcquote{Anonymous}{%
    Sure, I know the methods I've used in my papers, but I don't feel 
    particularly like a scientist.%
  }
\end{frame}

\begin{frame}
  \begin{center}
    What makes my work scientific?
  \end{center}
\end{frame}

\begin{frame}
  \fullcite{SecurityAsAScience}
\end{frame}

\begin{frame}
      \textcquote[\S IV]{SecurityAsAScience}{\textins*{C}laims of necessary 
        conditions for real-world security are unfalsifiable.
      Claims of necessary conditions for formally-defined security are 
    tautological restatements of the assumptions}.
\end{frame}

\begin{frame}
  \begin{remark}[This is a problem]
    \begin{itemize}
      \item The community itself is in disagreement on the Science of 
        Security~\cite{SecurityAsAScience}.
      \item This will be very confusing when entering the field.
    \end{itemize}
  \end{remark}

  \pause

  \begin{example}[According to some]
    \begin{itemize}
      \item Cryptography isn't science.
      \item But provable security is.
    \end{itemize}
  \end{example}
\end{frame}

\begin{frame}
  \textcquote{SecurityAsAScience}{%
    Unfalsifiable claims are common in security---and they, along with circular 
    arguments, are used to justify many defensive measures \textelp{}
    \textins{T}here are many ways to argue measures in, but no way to argue one 
    out.
  }
\end{frame}


\subsection{The goal}

\begin{frame}
  \begin{block}{The goal}
    \begin{itemize}
      \item Give a holistic view of Science of Security.

        \pause

      \item Where are the disputes, philosophical problems and why?
      \item What are the methods we use and why?
    \end{itemize}
  \end{block}
\end{frame}

\begin{frame}
  \begin{example}[\enquote{Provable security}]
    \begin{itemize}
      \item A uniformly random string of length \(n\) is the most secure 
        password.

      \item We can prove it will take millions of years to guess it.
    \end{itemize}
  \end{example}

  \pause

  \begin{remark}
    \begin{itemize}
      \item Attackers still get in, strange.
    \end{itemize}
  \end{remark}
\end{frame}

\begin{frame}
  \begin{example}[Usability]
    \begin{itemize}
      \item Turns out people can't handle uniformly random passwords.
      \item With a unique such password for every service.
    \end{itemize}
  \end{example}
\end{frame}



\section{Concrete suggestion}

\subsection{Contents}

\begin{frame}
  \begin{block}{Contents, part I}
    \begin{enumerate}
      \item Philosophy of Science of Security
      \item Purely deductive methods
      \item[\vdots]
      \item[n] Purely inductive methods
    \end{enumerate}
  \end{block}

  \pause

  \begin{remark}[What to focus]
    \begin{itemize}
      \item What makes a method scientific?
      \item How do these play together? (The holistic aspect.)
      \item Emphasize the deduction/induction divide~\cite{SecurityAsAScience}.
    \end{itemize}
  \end{remark}
\end{frame}

\begin{frame}
  \begin{example}[Philosophy of Science of Security]
    \begin{itemize}
      \item Start with a discussion of 
        \citetitle{SecurityAsAScience}\footfullcite{SecurityAsAScience}.
      \item What is Science of Security?
      \item Does that even exist at the moment?
      \item Shall we work according to the hypothetico-deductive model?
      \item What are the problems?
    \end{itemize}
  \end{example}
\end{frame}

\begin{frame}
  \begin{example}[Deductive inquiry]
    \begin{itemize}
      \item \enquote{What can deduction possibly say about reality?}
    \end{itemize}
  \end{example}

  \pause

  \begin{example}[Inductive/empirical inquiry]
    \begin{itemize}
      \item \enquote{What can induction possibly say about security?}
    \end{itemize}
  \end{example}

  \pause

  \begin{remark}[To focus on]
    \begin{itemize}
      \item What are the limitations?
      \item Do these require a combination to form a Science of Security?
    \end{itemize}
  \end{remark}
\end{frame}

\begin{frame}
  \begin{block}{Contents, part II}
    \begin{itemize}
      \item General introductions to various subfields.
      \item Which methods are used and why?
      \item Some exemplary papers?
      \item How does a subfield fit into the holistic picture of Security?
    \end{itemize}
  \end{block}
\end{frame}

\begin{frame}
  \begin{remark}
    \begin{itemize}
      \item All above was top down: faculty\footnote{%
          From different subfields.
        } present their view on
        \begin{itemize}
          \item the methodologies,
          \item the practices,
          \item the adversary models,
          \item the assumptions,
          \item the relation to scientific approach in their respective 
            subfield.
        \end{itemize}
    \end{itemize}
  \end{remark}
\end{frame}

\begin{frame}
  \begin{block}{Bottom up}
    \begin{itemize}
      \item Course participants review the scientific merits of 
        papers\footnote{%
          Chosen by subfield designer, not participants.
        } from top conferences in the subfield.

      \item They identify/reverse engineer methodology and components of 
        evaluation.

      \item They value why this is scientific and how and what knowledge it 
        contributes.
    \end{itemize}
  \end{block}
\end{frame}

\begin{frame}
  \begin{block}{Learning objectives}
    Should be able to
    \begin{itemize}
      \item choose an appropriate method of inquiry to answer a given research 
        question in the field of Security.

      \item assess how a paper contributes to the advancement of the field of 
        Security.

      \item evaluate the choice of methodology in a given paper.
    \end{itemize}
  \end{block}
\end{frame}

\begin{frame}[allowframebreaks]
  \begin{block}{Assessment}
    \begin{itemize}
      \item Apply subfield methodology insights to own paper.

      \item Reflect on how this paper fits in the big picture of Security as a 
        Science.

      \item Discussion/reflection on limits of how scientific security research 
        can be; \eg provability versus complexity of actual systems, 
        engineering versus science.

      \item Peer-review (among course participants) these individual 
        papers\footnote{%
          Or a paper in progress or already published paper.
        } to identify gaps in the scientific approach that could be filled.
    \end{itemize}
  \end{block}
\end{frame}

\subsection{Format}

\begin{frame}
  \begin{idea}
    \begin{itemize}
      \item Develop material jointly.
      \item Design as MOOC/asynchronous.
      \item This allows us to
        \begin{itemize}
          \item each run the course locally when we have new students, or
          \item run it jointly in relation to the SWITS seminar, and
          \item reuse parts of the material in other courses too.
        \end{itemize}
    \end{itemize}
  \end{idea}
\end{frame}

\begin{frame}
  \begin{block}{Teaching material}
    \begin{itemize}
      \item Develop material jointly (video lectures, exercises \etc).
      \item Each research group is specialized on a part of the Science of 
        Security methodology.

      \item Use tools that bridge the social aspects over time and space:
        \eg Perusall.
    \end{itemize}
  \end{block}
\end{frame}

\begin{frame}
  \begin{block}{Giving the course}
    \begin{enumerate}
      \item Give the course in relation to SWITS every year.
      \item Each faculty member can do assessment of their students locally, 
        \ie give it any time.
    \end{enumerate}
  \end{block}

  \pause

  \begin{remark}[Administration]
    \begin{enumerate}
      \item One host institution, others do credit transfer?
      \item Each institution has their own syllabus, course code \etc?
      \item Split into small modules, different institutions responsible for 
        each?
    \end{enumerate}
  \end{remark}
\end{frame}


\section{Discussion}

\begin{frame}
  \begin{center}
    Comments, questions, other thoughts?
  \end{center}
\end{frame}


\subsection{Summary}

\begin{frame}
  \begin{itemize}
    \item MOOC can be heavy to develop.
    \item Emphasis on practice: why are methods working, why can't we say more 
      about a result?
    \item How is the progression compared to the \enquote{normal} methodology  
      course?
      \begin{itemize}
        \item What is the added value?
      \end{itemize}
    \item Target audience? Half-way through students?
    \item 20+ students said they wanted such a course.
  \end{itemize}
\end{frame}


%%% REFERENCES %%%

\begin{frame}[allowframebreaks]
  \printbibliography
\end{frame}

