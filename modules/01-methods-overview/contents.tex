% Since this a solution template for a generic talk, very little can
% be said about how it should be structured. However, the talk length
% of between 15min and 45min and the theme suggest that you stick to
% the following rules:  

% - Exactly two or three sections (other than the summary).
% - At *most* three subsections per section.
% - Talk about 30s to 2min per frame. So there should be between about
%   15 and 30 frames, all told.


\section{The scientific method}

\subsection{Deduction}

\begin{frame}
  \begin{itemize}
    \item We have two major schools of gaining new knowledge: deduction and 
      induction.
    \item Both are used, but have different strengths and weaknesses.
  \end{itemize}
\end{frame}

\begin{frame}
  \begin{definition}[Deduction]
    \begin{itemize}
      \item This stems from logic and mathematics.
      \item State axioms.
      \item Use the axioms and formal rules (logic) to infer knowledge.
    \end{itemize}
  \end{definition}
\end{frame}

\begin{frame}
  \begin{itemize}
    \item In security, this is usually done with a \emph{system model} and an 
      \emph{adversary model}.
  \end{itemize}

  \pause

  \begin{definition}[System and adversarial model]
    \begin{itemize}
      \item The system model captures the core properties of the system.
      \item The adversary model captures the capabilities of the adversary.
    \end{itemize}
  \end{definition}

  \pause

  \begin{itemize}
    \item We define the security properties in terms of the system model.
    \item We use these models as axioms and try to show security deductively.
  \end{itemize}
\end{frame}

\begin{frame}
  \begin{example}[A simple model]
    \begin{itemize}
      \item There is a channel through which to communicate with Alice.
      \item Alice does not know who is on the other end.
      \item The adversary cannot eavesdrop on the channel and is 
        computationally limited\footnote{\Ie a polynomial-time Turing 
          machine.}.
    \end{itemize}
  \end{example}

  \begin{example}[Security properties]
    \begin{itemize}
      \item Alice believes she is talking to Bob \emph{if and only if} she is 
        in fact talking to Bob.
    \end{itemize}
  \end{example}
\end{frame}

\begin{frame}
  \begin{example}[Password-based authentication]
    \begin{itemize}
      \item Bob says (through the channel): \enquote{Hey, it's me, Bob!}.
      \item Alice replies: \enquote{Prove it!}.
      \item Bob says: \enquote{My password is p4ssw0rd}.
    \end{itemize}
  \end{example}

  \pause

  \begin{question}
    \begin{enumerate}
      \item Is this protocol secure in our model?
      \item Is this protocol secure in reality? (\Ie is it a good model?)
    \end{enumerate}
  \end{question}
\end{frame}

\begin{frame}
  \begin{remark}
    \begin{itemize}
      \item Cryptography is the area which uses most stringently the deductive 
        method.
      \item In many areas of security it is not possible use the full strict 
        formalism required in maths.
      \item To do deduction one needs simple building blocks with few 
        properties.
      \item Thus, in many papers, it's informal, rather than, formal arguments 
        in the deduction.
      \item And occasionally, we find vulnerabilities.
    \end{itemize}
  \end{remark}
\end{frame}

\begin{frame}
  \begin{remark}
    \begin{itemize}
      \item Cryptography with it's perfect deduction isn't problem free either.
      \item The axioms must match the real world, which is hard.
      \item Sometimes things are over-simplified.
    \end{itemize}
  \end{remark}
\end{frame}

\subsection{Induction}

\begin{frame}
  \begin{definition}[Induction, hypothesis testing]
    \begin{enumerate}
      \item\label{FormHypothesis} Form hypothesis.
      \item Perform experiment and collect data.
      \item Analyse data.
      \item Interpret data and draw conclusion.
      \item Depending on conclusions, return to~\ref{FormHypothesis}.
    \end{enumerate}
  \end{definition}
\end{frame}

\begin{frame}
  \begin{remark}
    \begin{itemize}
      \item There are requirements on the induction process.
      \item The experiment must test a hypothesis which is both \emph{testable} 
        and \emph{falsifiable}.
      \item The experiment must have exactly \emph{one variable}.
      \item The experiment must be \emph{reproducible} and \emph{results 
          repeatable}.
    \end{itemize}
  \end{remark}
\end{frame}

\begin{frame}
  \begin{example}[Testable and falsifiable]
    \begin{itemize}
      \item Implies observability and measurability.
      \item Hypothesis: Prayer increases contact with one's deity.
      \item Not falsifiable: We cannot observe divine communication, no 
        experiment could prove the result wrong.
    \end{itemize}
  \end{example}

  \pause

  \begin{example}[Testable and falsifiable]
    \begin{itemize}
      \item Hypothesis: This software is secure.
      \item Not measurable: What is secure?
    \end{itemize}
  \end{example}
\end{frame}

\subsection{Quantitative and qualitative methods}

\begin{frame}
  \begin{definition}[Quantitative methods]
    \begin{itemize}
      \item Explores things that can be objectively quantified.
      \item \Eg measuring time, length, weight, cardinality.
    \end{itemize}
  \end{definition}

  \begin{definition}[Qualitative methods]
    \begin{itemize}
      \item Explores things that are qualitative (subjective) --- cannot be 
        quantified.
      \item Used to explore phenomena in-depth.
      \item Especially useful for finding direction in usability research.
    \end{itemize}
  \end{definition}
\end{frame}

\begin{frame}
  \begin{example}
    \begin{itemize}
      \item We've designed two interfaces and want to compare how good they 
        are.

      \item \emph{Quantitatively}: we measure the time it takes for people to 
        finish a task in both interfaces.

      \item \emph{Qualitatively}: we interview people about what they think of 
        each of the interfaces.
    \end{itemize}
  \end{example}
\end{frame}

\begin{frame}
  \begin{remark}
    \begin{itemize}
      \item The quantitative results allow us to determine which of the two 
        interfaces is the best.

      \item The qualitative results allows us to extract what is good about the 
        interfaces.

      \item Thus we can potentially construct a third, much better, interface.
    \end{itemize}
  \end{remark}
\end{frame}



%%%%%%%%%%%%%%%%%%%%%%

\begin{frame}[allowframebreaks]
  \small
  \printbibliography
\end{frame}

