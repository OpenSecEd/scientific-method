\title{%
  How do you know it's secure?\\
  Password policies
}
\author{Daniel Bosk\thanks{%
    This material is available under the Creative Commons 
    Attribution-ShareAlike (CC-BY-SA) 4.0 international license.
    The material was written with the aid of GitHub Copilot.
}}
\institute{%
  KTH EECS
}

\begin{frame}
  \maketitle
\end{frame}

\mode*

\begin{abstract}
  \mode*

% What's the problem?
% Why is it a problem? Research gap left by other approaches?
% Why is it important? Why care?
% What's the approach? How to solve the problem?
% What's the findings? How was it evaluated, what are the results, limitations, 
% what remains to be done?

% XXX Summary
\emph{Summary:}
In this learning session we will give an introduction to the scientific method 
and particularly how this can be applied in the area of security.

% XXX Motivation and intended learning outcomes
\emph{Intended learning outcomes:}
After this session you should be able:
\begin{itemize}
  \item to \emph{apply} the scientific method correctly to answer basic 
    questions in security.
\end{itemize}

% XXX Prerequisites
%\emph{Prerequisites:}
%\dots

% XXX Reading material
\emph{Reading:}
You should read 
\citetitle{HowToDesignSecurityExperiments}~\cite{HowToDesignSecurityExperiments}.
This paper discusses the scientific method of (parts of) the security field.
For a more in-depth reflection on the state of security as a scientific pursuit, 
we recommend
\citetitle{SecurityAsAScience}~\cite{SecurityAsAScience}.

\end{abstract}

\clearpage

\section{Introduction}

\begin{frame}
We've had passwords for about as long as we've had computers.
Unfortunately, we still\footnote{As of \today.} haven't managed to do it right 
in practice.
\end{frame}
So here we'll deal with the following question.

\begin{frame}
  \begin{question}\label{RQ}
    How can we know how secure our password-based authentication system will be?
  \end{question}
\end{frame}


\section{How do we know it's secure?}

\begin{frame}
We have a system where users log in.
We want to authenticate the users securely.
We've decided to use a password-based authentication system\footnote{%
  Yes, I know it's a bit of an oxymoron, but humor me.
}.
\end{frame}

\begin{frame}
  \begin{question}
    What do we need to know to try to answer \cref{RQ}?
  \end{question}
\end{frame}


\subsection{What do we mean by secure?}

Well, first of all, we need to define what we mean by \enquote{being secure}.
\Cref{RQ} asks us to estimate how secure a password-based authentication system 
is.
That means that there might be several possible levels of security.

\begin{frame}[fragile]
  \begin{question}
    How do we define security for an authentication system?
  \end{question}
\end{frame}

\begin{frame}
\begin{definition}[Informal definition of security]
  The authentication system is \emph{secure} if it is \emph{hard} for an 
  adversary to authenticate a false claim.
  We let the \emph{security level} of an authentication system be the inverse 
  probability of a successful attack.
\end{definition}
\end{frame}

We can also define security in terms of the amount of work an adversary has to.
