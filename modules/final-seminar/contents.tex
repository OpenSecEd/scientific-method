\title{%
  Assessment:\\
  Designing a methodology to answer a question
}
\author{Daniel Bosk\thanks{%
   This material was authored by Daniel Bosk and is available under the 
   Creative Commons Attribution-ShareAlike (CC-BY-SA) 4.0 international 
   license.
   GitHub Copilot was used to autocomplete parts of the source code of this 
   document.
}}
\institute{%
  KTH EECS
}

\mode*

\begin{frame}
  \maketitle
\end{frame}

\begin{abstract}
  \mode*

% What's the problem?
% Why is it a problem? Research gap left by other approaches?
% Why is it important? Why care?
% What's the approach? How to solve the problem?
% What's the findings? How was it evaluated, what are the results, limitations, 
% what remains to be done?

% XXX Summary
\emph{Summary:}
In this learning session we will give an introduction to the scientific method 
and particularly how this can be applied in the area of security.

% XXX Motivation and intended learning outcomes
\emph{Intended learning outcomes:}
After this session you should be able:
\begin{itemize}
  \item to \emph{apply} the scientific method correctly to answer basic 
    questions in security.
\end{itemize}

% XXX Prerequisites
%\emph{Prerequisites:}
%\dots

% XXX Reading material
\emph{Reading:}
You should read 
\citetitle{HowToDesignSecurityExperiments}~\cite{HowToDesignSecurityExperiments}.
This paper discusses the scientific method of (parts of) the security field.
For a more in-depth reflection on the state of security as a scientific pursuit, 
we recommend
\citetitle{SecurityAsAScience}~\cite{SecurityAsAScience}.

\end{abstract}

\clearpage


\section{Introduction}

Now you should be able to design a method to answer a given research question.
Hence, in this assignment you should show that you can do exactly that.
You will be assigned a research question for which you should propose a way to 
answer scientifically.

\begin{frame}<presentation>
  \begin{block}{Goal}
    \begin{itemize}
      \item You should be able to design a method to answer a research 
        question.
      \item Security is multifaceted, you should be able to consider more than 
        one perspective.
      \item You should be able to ask good questions and know how to answer 
        them.
    \end{itemize}
  \end{block}
\end{frame}

%Combine different methods to answer.
%Supply chain.
%Secure micro-services.
%
%Feedback to improve method for another question.
%
%Background: formal verification etc., different approaches.
%
%Inspelad presentation eller live. 4h x 4 i tentaperioden.

\section{Scenario and research question}

The research question that you should answer is the following:
\begin{frame}
\begin{question}
  We want to develop a secure instant-messaging system\footnote{%
    Think chat or SMS/text messages; something like Signal, WhatsApp, Telegram, 
    \etc
  }.
  How can we evaluate its security?
\end{question}
\end{frame}

\section{Assessment}\label{Assessment}

In brief, what you should be able to do is to ask good questions and propose 
suitable ways of answering them.
More formally, the learning objectives that we want to assess are the 
following; after completion of the course we want the student to be able to
\begin{enumerate}[label={(LO\arabic*)},ref=LO\arabic*]
  \item \LOrelate*;
  \item \LOevaluate*;
  \item \LOapply*;
  \item \LOplan*; and
  \item \LOcomm*.
\end{enumerate}

\subsection{Material you should produce}

To be able to show that you can do that, you should
\begin{frame}
  \only<presentation>{You should}
\begin{itemize}
  \item write a report, and
  \item make a presentation (10 minutes).\footnote{%
      If you write your report in LaTeX, consider writing your slides in LaTeX 
      also using the \texttt{beamer} and \texttt{beamerarticle} packages.
    }
\end{itemize}
These should should contain the following sections:
\begin{enumerate}
  \item Research question overview.
    \only<article>{This section should give an overview of the research 
      question and outline any subquestions that you derive.
    (This focuses on \cref{LOrelate}.)}
  \item Methodology.
    \only<article>{This section contains how you propose to answer the 
      questions from the previous section (methods used).
    (This focuses on \cref{LOapply}.)}
  \item Discussion.
    \only<article>{In this section you discuss why those methods answer the 
      questions properly and any limitations that you see.
      You can also discuss item alternative methods that you discarded (and 
      why).
    (This focuses on \cref{LOevaluate}.)}
  \item Conclusion.
    \only<article>{This section ties the sack.
      Here you connect the questions and the types of answers gained (through the 
      methods) and piece them back into the original research question.
      You also summarize how well you find the original question to be answered, 
      if there are any \enquote{holes that need filling}.
    (This focuses on \cref{LOrelate,LOevaluate,LOcomm}.)}
%  \item your original plan for the course work, adaptations made and what you 
%    learned.
\end{enumerate}
\end{frame}
This assesses \cref{LOrelate,LOevaluate,LOapply,LOcomm,LOplan}.

%\paragraph{Feedback you should provide}
%
%You should also
%\begin{itemize}
%  \item review someone else's report.
%\end{itemize}
%This assesses \cref{LOrelate,LOevaluate} and provides a learning opportunity 
%for \cref{LOcomm}.
%This also contributed to \cref{LOplan} as one would learn from others' 
%mistakes.
%
%The presentation should also include
%\begin{itemize}
%  \item what you learned from the feedback of the reviewer, and
%  \item what you learned from the work you reviewed.
%\end{itemize}
%This also assesses \cref{LOcomm}.

\subsection{Assessment criteria}

To assess the learning objectives
(\cref{LOrelate,LOevaluate,LOapply,LOplan,LOcomm})
we use the following criteria.
These criteria are also included as a rubric in the assignment where you hand 
in your report.

We note, however, that there is no criteria for \cref{LOplan}.
This is because we do not assess the plan itself, but rather the fact that you 
finished on time.

You'll need a pass on all criteria to pass the assignment and the course.

\begin{frame}[fragile,allowframebreaks]
  \RaggedRight
  \begin{longtable}
  {p{0.33\textwidth}p{0.33\textwidth}p{0.33\textwidth}p{0.33\textwidth}}
  \toprule
  \textbf{Learning objective}
    & \textbf{Criteria}
    & \textbf{Pass}
    & \textbf{Fail}
    \\*
  \midrule
  \endhead
  \only<article>{\cref{LOrelate}:}
  The student is able to \LOrelate
    & The main research question is explored from relevant aspects?
    & There might be more aspects to explore, but the most important ones are 
    covered. Motivate why no more aspect need to be explored.
    & There is at least one aspect missing that can be motivated to be 
    important. Motivate which one.
    For instance, do we need to ask another more detailed (research) question 
    to be able to answer the main research question in a meaningful way? Do 
    they address the question from just a single perspective?
    \\*
  \newpage
  \only<article>{\cref{LOapply}:}
  The student is able to \LOapply
    & The methods are suitable to answer the questions?
    & All questions have suggested methods that can actually answer the 
    question correctly.
    Motivate why this is the case.
    & There is at least one question that will not be answered correctly with 
    the suggested method.
    State which one and why.
    For instance, the method might only answer part of the question.
    %Or not at all.
    \\*
  \newpage
  \only<article>{\cref{LOevaluate}:}
  The student is able to \LOevaluate
    & Are all quality aspects considered in the discussion?
    & The most important quality aspects are considered and discussed.
    & At least one important quality aspect is missing.
    State which one and motivate why it's important enough that it must be 
    treated.
    \\*
    & Are all ethical aspects considered?
    & The most important ethical aspects are considered and discussed.
    & At least one important ethical aspect is missing.
    State which one and motivate why it's important enough that it must be 
    treated.
    \\*
  \newpage
  \only<article>{\cref{LOcomm}:}
  The student is able to \LOcomm
    & Is the report written as short as possible?
    & The report can probably be slightly shortened, but not by much.
    & The report can be shortened considerably.
    Give at least one example of where and how.
    \\*
    & Is the report clear and easy to understand?
    & The report is easy to understand.
    & Some parts of the report must be read more than once to understand.
    (Or worse.)
    Give at least one example.
    \\*
    & Are the arguments clearly stated and well motivated?
    & All arguments are clearly stated and well motivated.
    & At least one argument is not clearly stated or not well motivated.
    State which one and motivate why it's not clear or well motivated.
    \\*
  \bottomrule
  \end{longtable}
\end{frame}

\only<article>{%
\subsection{Plagiarism}

You work in the groups that you've signed up for.
You may discuss with others, search the literature and use tools such as 
ChatGPT as long as you are transparent with its use:
\ie you cite sources and state what you've used ChatGPT for\footnote{%
  For instance, if you've asked it for suggestions on how to shorten a piece of 
  text, you state that you did and how you used its suggestion.
}.

The governing rules that apply are from Chapter 10 of the Higher Education 
Ordinance, from its \S~1:
\begin{quote}
  Disciplinary measures may be taken against students who
  \begin{enumerate}
    \item use prohibited aids or other methods to \emph{attempt to deceive} 
      \textins{my emphasis} during examinations or other forms of assessment of 
      study performance,
  \end{enumerate}
\end{quote}

Not mentioning that you've used ChatGPT or discussed the topic with people 
outside the group is considered an attempt to deceive.
}

\section{The final seminar}

\subsection{Structure}

\begin{frame}
  \begin{block}{Structure}
    \begin{enumerate}
      \item Each group presents
      \item After each presentation, I might ask for some more details.
      \item When everyone has presented, we discuss what we've heard.
    \end{enumerate}
  \end{block}
\end{frame}

\subsection{Questions for discussion}

\begin{frame}
  \begin{question}
    What are the first thoughts while/after hearing the others present?
  \end{question}
  \begin{question}
    What did you like the most from what you've heard?
  \end{question}
  \begin{question}
    What would you like to improve/do differently/add to your own work 
    after hearing what the others did?
  \end{question}
\end{frame}
