\emph{Summary:}
The course highlights, how different parts of the scientific methodology are 
relevant for cybersecurity in different situations. The focus is to analyse how 
scientific methods influence our knowledge of issues in cybersecurity, 
including their relation to different aspects of the subjects
\begin{itemize}
  \item gender equality, diversity and equal conditions
  \item sustainability,
  \item ethical dilemmas.
\end{itemize}

\emph{Intended learning outcomes:}
After passing the course, the student should be able to
\begin{itemize}
  \item \emph{relate} the different parts of scientific method, how they relate 
    to one another, contribute and not contribute to scientificity in security;
  \item \emph{assess, analyse, and discuss} the quality in, and ethical aspects 
    of, knowledge generation related to digital systems and in particular the 
    security of these systems;
  \item \emph{apply} scientific methodology to show how to answer issues in the 
    cybersecurity field;
  \item \emph{plan and carry out} assignments within given time frames and 
    available resources;
  \item \emph{write} short, clear and \emph{arguing} texts based on \emph{own 
    analysis} as well as given material
\end{itemize}
in order to be able to contribute to scientifically based development.

\emph{Prerequisites:}
Knowledge and skills in cyber security, the student should be able to:
\begin{itemize}
  \item identify threats against confidentiality, integrity and availability in 
    digital systems
  \item explain basic terminology and concepts in computer security and use 
    them
  \item find and use documentation of security related problems and tools
  \item analyse simple program code and systems (based on given or self-made 
    system descriptions) to identify vulnerabilities and predict corresponding 
    threats
  \item select countermeasures against identified threats and argue for their 
    suitability
  \item compare countermeasures and evaluate their side effects,
  \item apply countermeasures present and explain their reasoning to others.
\end{itemize}

Knowledge of the role of the cybersecurity engineer in society, the student 
should be able to:
\begin{itemize}
  \item analyse and discuss how the use and development of digital systems and 
    in particular the security of these systems affect and are affected by 
    social, economic, environmental, work environmental and ethical 
    sustainability as well as diversity, gender equality and equal conditions
  \item review critically and reflect on both the set-up and implementation of 
    the education as well as their own study situation, their skills in 
    relation to the objective of the education and the future professional role 
    and their ability to identify their own need of additional knowledge
  \item plan and carry out assignments within given time frames and using 
    available resources
  \item write short, clear and arguing texts based on own analysis as well as 
    given material.
\end{itemize}

Knowledge of scientific methodology, the student should, with regards to the 
theory and  methodology of science, both orally as well as in writing, be able 
to:
\begin{itemize}
  \item identify definitions and descriptions of concepts, theories and problem 
    areas, as well as identify the correct application of these concepts and 
    theories;
  \item account for concepts, theories and general problem areas, as well as 
    apply concepts and theories to specific cases;
  \item critically discuss the definitions and applications of concepts and 
    theories as they applies to specific cases of scientific research.
\end{itemize}

% XXX Assessment
\emph{Assessment:}
The intended learning outcomes are assessed in the form of written assignments 
(asynchronous) and active seminar participation (synchronous).
