\documentclass[a4paper,10pt,article,oneside]{memoir}
%%% Tufte %%%
\usepackage{marginfix}
%\setlength{\evensidemargin}{\oddsidemargin}
\marginparmargin{outer}
\setlrmarginsandblock{2.5cm}{8cm}{*}

\footnotesinmargin

\usepackage{ragged2e}
\renewcommand{\sidefootform}{\RaggedRight}
\renewcommand{\foottextfont}{\footnotesize\RaggedRight}

\setmpjustification{\RaggedRight}{\RaggedRight}

% margin figure and caption typeset ragged against text block
\setfloatadjustment{marginfigure}{\mpjustification}
\setmarginfloatcaptionadjustment{figure}{\captionstyle{\mpjustification}}

% From https://tex.stackexchange.com/a/324757/17418
% Palatino for main text and math
\usepackage[osf,sc]{mathpazo}

% Helvetica for sans serif
% (scaled to match size of Palatino)
\usepackage[scaled=0.90]{helvet}

% Bera Mono for monospaced
% (scaled to match size of Palatino)
\usepackage[scaled=0.85]{beramono}

\setlxvchars\setxlvchars
\checkandfixthelayout

\nouppercaseheads
%%% end tufte %%%
\let\subsubsection\subsection
\let\subsection\section
\let\section\chapter

\newsubfloat{figure}% Allow subfloats in figure environment

\usepackage[utf8]{inputenc}
\usepackage[T1]{fontenc}
\usepackage[swedish,british]{babel}
\usepackage[strict]{csquotes}
\usepackage{url}
\usepackage[capitalize]{cleveref}
\usepackage{booktabs}
\usepackage{graphicx}
\usepackage{color}
\usepackage{keystroke}

\usepackage[all]{foreign}
\renewcommand{\foreignfullfont}{}
\renewcommand{\foreignabbrfont}{}

\usepackage[natbib,style=alphabetic,maxbibnames=99]{biblatex}
\addbibresource{scientific-method.bib}

\usepackage[single]{acro}

\usepackage{subcaption}

\usepackage[noend]{algpseudocode}
\usepackage{xparse}

\usepackage{listings}
\lstset{%
  basicstyle=\footnotesize,
  numbers=left
}

\usepackage{amsmath}
\usepackage{amssymb}
\usepackage{mathtools}
\usepackage{amsthm}
\usepackage{thmtools}
\usepackage[unq]{unique}

\usepackage[binary-units]{siunitx}


\usepackage[noamsthm,notheorems]{beamerarticle}
\setjobnamebeamerversion{slides}

\usepackage[inline]{enumitem}

\usepackage{authblk}
\let\institute\affil

\declaretheorem[numbered=unless unique,style=theorem]{theorem}
\declaretheorem[numbered=unless unique,style=definition]{definition}
\declaretheorem[numbered=unless unique,style=definition]{assumption}
\declaretheorem[numbered=unless unique,style=definition]{protocol}
\declaretheorem[numbered=unless unique,style=example]{example}
%\declaretheorem[style=definition,numbered=unless unique,
%  name=Example,refname={example,examples}]{example}
\declaretheorem[numbered=unless unique,style=remark]{remark}
\declaretheorem[numbered=unless unique,style=remark]{idea}
\declaretheorem[numbered=unless unique,style=exercise]{exercise}
\declaretheorem[numbered=unless unique,style=exercise]{question}
\declaretheorem[numbered=unless unique,style=solution]{solution}

\begin{document}
\title{%
  How do you know it's secure?
  Passwords
}
\author{Daniel Bosk\thanks{%
    This material is available under the Creative Commons 
    Attribution-NonCommercial-ShareAlike (CC-BY-NC-SA) 4.0 international 
    license.
    The material was written with some aid from GitHub Copilot.
}}
\institute{%
  KTH EECS
}

\begin{frame}
  \maketitle
\end{frame}

\mode*

\begin{abstract}
  \mode*

% What's the problem?
% Why is it a problem? Research gap left by other approaches?
% Why is it important? Why care?
% What's the approach? How to solve the problem?
% What's the findings? How was it evaluated, what are the results, limitations, 
% what remains to be done?

% XXX Summary
\emph{Summary:}
In this learning session we will give an introduction to the scientific method 
and particularly how this can be applied in the area of security.

% XXX Motivation and intended learning outcomes
\emph{Intended learning outcomes:}
After this session you should be able:
\begin{itemize}
  \item to \emph{apply} the scientific method correctly to answer basic 
    questions in security.
\end{itemize}

% XXX Prerequisites
%\emph{Prerequisites:}
%\dots

% XXX Reading material
\emph{Reading:}
You should read 
\citetitle{HowToDesignSecurityExperiments}~\cite{HowToDesignSecurityExperiments}.
This paper discusses the scientific method of (parts of) the security field.
For a more in-depth reflection on the state of security as a scientific pursuit, 
we recommend
\citetitle{SecurityAsAScience}~\cite{SecurityAsAScience}.

\end{abstract}

\clearpage

\section<article>{Introduction}

We've had passwords for about as long as we've had computers.
Unfortunately, we still\footnote{As of \today.} haven't managed to do it right 
in practice.
So here we'll deal with the following question.

\begin{question}\label{RQ}
  How can we know how secure our password-based authentication system will be?
\end{question}


\section[How do we know?]{How do we know it's secure?}

\begin{frame}
We have a system where users log in.
We want to authenticate the users securely.
We've decided to use a password-based authentication system\footnote{%
  Yes, I know it's a bit of an oxymoron, but humor me.
}.

  \begin{exercise}
    What do we need to know to try to answer
    \only<article>{\cref{RQ}}%
    \only<presentation>{how secure this will be}%
    ?
  \end{exercise}
\end{frame}


\section[Define secure?]{What do we mean by secure?}

Well, first of all, we need to define what we mean by \enquote{being secure}.
\Cref{RQ} asks us to estimate how secure a password-based authentication system 
is.
That means that there might be several possible levels of security.

\begin{frame}[fragile]
  \begin{exercise}
    How do we define security for an authentication system?
  \end{exercise}
\end{frame}

\begin{frame}
  \begin{solution}
    The first thing to do is to investigate how others have defined this.
    So our approach will be to do a literature review.
  \end{solution}
\end{frame}

\subsection{Literature reviews}

There are several ways to do a literature review, or literature study.
The first, and more rigorous, is to do a systematic literature review.
The second, is a more informal approach.
Which one to use depends on the purpose.

\paragraph{Systematic literature review}

Let's start with the systematic literature review.
\blockcquote{ANUSLR}{%
The broad purposes of a systematic literature review are to categorise the 
literature systematically, and to make it transparent how the researcher found, 
selected and assessed the literature.%
}
If that is important, we should do a systematic literature review.

In a systematic literature review, we will document the 
following\autocite{ElsevierSLR}:
\begin{itemize}
  \item Eligibility criteria for included research.
    \Eg that the research is peer reviewed, that the abstract makes it 
    obvious it's about authentication.
  \item A description of the systematic research search strategy.
    Do we search in Google Scholar, Scopus, Web of Science, or some other 
    database?
  \item An assessment of the validity of reviewed research.
  \item Interpretations of the results of research included in the review.
  \item Among other details.
\end{itemize}
The main aim is that someone else should be able to do the exact same thing and 
arrive at the exact same results.
But perhaps more interestingly, one should be able to redo it later to see the 
changes, what has happened since the last time it was done.

\paragraph{Literature review}

Sometimes that level of detail and reproducibility is not important.
For instance, if we just want to see a few different common definitions for 
inspiration, then we can do a (non-systematic) literature review.

In the case of a literature review, we still search the scientific literature.
However, we don't need to document systematically how we do it.

\begin{frame}
  \begin{exercise}
    We want to investigate \alert<2>{the most common} definitions of security 
    for authentication systems and for which types of authentication systems 
    they're used.
    Which type of literature review should we do; systematic or non-systematic?
    Why?
  \end{exercise}

  \begin{onlyenv}<article>
In this case, the goal is to get an overview of the literature.
Since purpose is to research what definitions there are and how they're used, 
we must document our method of research.
This means that we should do a systematic literature review.
  \end{onlyenv}

  \begin{exercise}
    We want to \alert<2>{explore \only<2>{\textins{some} }different} 
    definitions of security for authentication systems to find a definition to 
    use for our study.
    Which type of literature review should we do; systematic or non-systematic?
    Why?
  \end{exercise}
\end{frame}

Well, if we know approximately what we want, then we can just browse parts of 
the literature until we've found what we want.
We don't need to document our method of research, since we can choose any 
definition of our liking or even make up our own.

However, it's worth noting that if someone else is using that definition, we 
should add a reference to it.
This reference serves two purposes:
\begin{enumerate}
  \item It gives attribution to whoever came up with the definition.
    Otherwise, we're plagiarizing.
  \item It shows that we're not the only ones using this definition.
    All works using the same definition can easily be compared.
\end{enumerate}

\paragraph{Case studies on literature reviews}

We'll study three papers on this topic in the coming assignments.

The first\autocite{GraphicalPasswordsSurvey} is a systematic literature review 
on the topic of graphical passwords (since that is closely related to our 
example case here).

The second\autocite{OfPasswordsAndPeople} is a normal research paper with a 
non-systematic literature review section.
The goal is to how it compares to the systematic literature review.

The third, and last, is a paper on how to do a systematic literature review in 
the area of information systems\autocite{SLRinIS}.
It covers the details of what to think about, but also discusses how to use 
tools to aid the process.

\subsection{A definition of security}

Let's consider our definition of security for an authentication system.
We know approximately what we want, we just want to explore different 
possibilities for the details.

\begin{frame}
\begin{definition}[Informal definition of security]
  The authentication system is \emph{secure} if it is \emph{hard} for an 
  adversary to authenticate a false claim.
  We let the \emph{security level} of an authentication system be the inverse 
  probability of a successful attack.
\end{definition}
\end{frame}

Now, that captures the essence of what we want.
However, we just reduced the problem of defining \enquote{being secure} to 
defining what \enquote{hard} means.
This leads us down the path of formal security, \eg using complexity theory.

\begin{exercise}
  Search for a suitable formal definition of security for an authentication 
  system, one that captures what we've laid out above.
\end{exercise}

When I did this, I first searched for authentication.
Then I tried to look for any definitions among the results.
I didn't find anything.
But it struck me that an authentication system is essentially the same as an 
(interactive) proof system in cryptography.
That was my next search and, inspired by the results, I settled on the 
following definitions\footnote{%
  \Cref{FormalSecurity} I made them up myself.
  \Cref{Negligible} is very standard, something with similar phrasing can be 
  found on Wikipedia even.
  Actually GitHub Copilot autocompleted \cref{Negligible} for me.
  I had to double check it to make sure it was actually what I intended.
  And it brings us into a copyright and plagiarism gray zone.
}.

\begin{frame}
\begin{definition}[Formal definition of security]\label{FormalSecurity}
  We have a prover~\(P\) and a verifier~\(V\).
  \begin{description}
    \item[Completeness] If the prover~\(P\) knows the secret, then the 
      verifier~\(V\) will accept with probability \(1\).
    \item[Soundness] If the prover~\(P\) does not know the secret, then the 
      verifier~\(V\) will accept with \emph{negligible} probability.
  \end{description}
\end{definition}

\begin{definition}[Negligible]\label{Negligible}
  A function \(f\) is \emph{negligible} if for all positive integers~\(c\) 
  there exists a positive integer~\(n_0\) such that for all \(n > n_0\) it 
  holds that \(f(n) < \frac{1}{n^c}\).
\end{definition}
\end{frame}

The ambition with these definitions is to capture that if we just have a 
complex enough password, the adversary will not be able to guess it.


\section{Evaluating security}

\begin{frame}
  \begin{exercise}
    How can we evaluate the security of our authentication system?
  \end{exercise}
\end{frame}

\begin{frame}
  \begin{solution}
    We must investigate how difficult the passwords are to guess 
    (\cref{FormalSecurity}).
    We can divide the path forward into two ways:
    \begin{enumerate}
      \item A deductive proof.
      \item An empirical investigation.
    \end{enumerate}
  \end{solution}
\end{frame}

\subsection{Deductive evaluation}

To do a deductive proof\footnote{%
  We have many courses dedicated to this:
  Introduced in DD2395 Computer Security,
  expanded in DD2520 Applied Cryptography,
  full immersion in DD2448 Foundations of Cryptography.
  And then we have the range of algorithms and complexity courses:
  \eg DD2350 Algorithms, Data Structures and Complexity.
}, we need to know something about the password distribution.
If some passwords are more likely than others, then the adversary can guess 
them more easily and we will not be able to fulfil \cref{FormalSecurity}.

\begin{frame}
  \begin{assumption}\label{AssumeUniform}
    The passwords are uniformly distributed.
  \end{assumption}
\end{frame}

A uniform distribution means that
all passwords are equally likely (\(\frac{1}{N^n}\))
and that
the Shannon entropy is maximized and equal to \(-\log \frac{1}{N^n} = n \log 
{N}\),
where \(N\) is the number of possible characters and \(n\) is the length of the 
password.

\begin{frame}
  \begin{proof}
    We see that the probability of guessing a password is \(\frac{1}{N^n}\).
    This is negligible (\cref{Negligible}), so we have soundness 
    (\cref{FormalSecurity}).
    (Completeness is straight forward, the verifier will accept if the password 
    is correct.)
  \end{proof}
\end{frame}

\begin{frame}
  \begin{exercise}
    We've proved it secure.
    Is it really secure, why or why not?
    How can we answer this question?
  \end{exercise}

  \begin{solution}
    We can try forcing some user-generated passwords.

    We've assumed that the passwords are uniformly distributed.
    But are they?
    How can we find out?
  \end{solution}
\end{frame}

\subsection{Empirical evaluation}

Turns out people don't pick passwords like they pick cryptographic keys.
So \cref{AssumeUniform} is an assumption that doesn't resemble the real world 
particularly well; so any results derived from it, will not say much about the 
real world either.
This brings us to the following question.

\begin{frame}
\begin{question}\label{PasswordDistribution}
  What is the password distribution?
  How are passwords chosen?
\end{question}
\end{frame}

Usually there is a \emph{password policy} which affects how users choose 
passwords, and consequently affects the password distribution.
So we should change the question into the following.

\begin{frame}
\begin{question}[Password distribution]
  How does different password policies affect the password distribution?
\end{question}
\begin{question}[Password distribution, guessability]\label{Guessability}
  How easily can we guess the passwords under different password policies?
\end{question}
\begin{exercise}
  How should we try to answer these questions?
\end{exercise}
\end{frame}

\paragraph{Case studies on empirical evaluation}

\Textcite{GuessingHumanChosenSecrets2012} spent quite some time exploring these 
questions (his PhD thesis).
But we also have the papers by
\textcite{OfPasswordsAndPeople} and
\textcite{CanLongPasswordsBeSecureAndUsable}.
We will explore these papers to see how they tried to answer these questions, 
so we'll return to them.

\begin{frame}<presentation>
  \begin{example}[Password distribution, guessability]
    \fullcite{OfPasswordsAndPeople}
  \end{example}

  \begin{example}[Guessability, usability]
    \fullcite{CanLongPasswordsBeSecureAndUsable}
  \end{example}
\end{frame}

However, we can do other estimates deductively too.
For instance, we have the very famous \enquote{correct horse battery staple} 
from xkcd (\cref{xkcd936}).
If we can estimate the password distribution by knowing the password policy.
However, we still have \cref{AssumeUniform}, it's just distributed uniformly 
within the bounds of the password policy.

\begin{frame}
  \begin{figure}[h]
    \includegraphics[width=\linewidth]{fig/password_strength.png}
    \caption{%
      The famous xkcd \enquote{correct horse battery staple} comic.
      Image: xkcd.com/936/.
    }\label{xkcd936}
  \end{figure}
\end{frame}

\begin{frame}<presentation>
  \begin{exercise}
    Did any of the papers answer the question of how the passwords are chosen?
  \end{exercise}
\end{frame}

\section[Is it a good model?]{But is it even a good model to begin with?}

\begin{frame}
  \begin{exercise}
    Is this a good model/definition of security?
    What questions should we ask?
    How can we answer them?
  \end{exercise}
\end{frame}

Well, our model says that the verifier is benign.
We have no requirement that the verifier should not be able to learn the 
password.
So we have to assume that the verifier will never act as an adversary.
This leads us to the following question.

\begin{frame}
  \begin{question}
    Can the verifier be an adversary or is the verifier always benign?
    What are the consequences of this?
  \end{question}
\end{frame}

\begin{frame}<presentation>
  \begin{example}
    \fullcite{WhyPhishingWorks}
  \end{example}

  \pause

  \begin{example}[Consequences]
    We need the zero-knowledge property in our security definition.
    (We actually need \emph{malicious}, not honest, verifier zero-knowledge.
  \end{example}
\end{frame}

Now this depends on the users.
Can they tell a benign verifier from an adversary?
Turns out they can't\autocite{WhyPhishingWorks}.
\Citeauthor{WhyPhishingWorks} actually found that users might actually test 
whether the verifier is benign or not by entering their password:
the reasoning was that if the is correct it will accept the password, otherwise 
it will not (since it doesn't know the password)---which is a fallacy.

We can also again turn to the wisdom of xkcd for another argument against the 
benign verifier assumption, namely password reuse (\cref{xkcd792}).

\begin{frame}
  \begin{figure}
    \includegraphics[height=0.9\textheight]{fig/password_reuse.png}
    \caption{%
      Illustrating whether the benign verifier assumption is a good idea in 
      practice.
      Image: xkcd.com/792/.
    }\label{xkcd792}
  \end{figure}
\end{frame}

\paragraph{Case study}

We will study the paper by \textcite{WhyPhishingWorks} to explore how they 
answered this question.
So we'll return to this paper later.


\section{Conclusion}

\begin{frame}
  We might need a qualitative (\eg usability) study
  \only<presentation>{\newline}%
  to inform our deductive (\eg cryptographic) choices.
\end{frame}

%\begin{frame}
%  \begin{question}
%    Are there more reasons?
%    Password re-use, incompetent service, malicious service.
%  \end{question}
%\end{frame}

\end{document}
